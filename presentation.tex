\documentclass[11pt]{beamer}
\usetheme{CambridgeUS}
\usepackage[utf8]{inputenc}
\usepackage{amsmath}
\usepackage{amsfonts}
\usepackage{multicol}
\usepackage{amssymb}
\usepackage{graphicx}
\author{Alexander Papiez}
\title{September Presentation}
%\setbeamercovered{transparent} 
%\setbeamertemplate{navigation symbols}{} 
%\logo{} 
%\institute{} 
\date{Friday 16$^{th}$ September, 2016} 
%\subject{}
%%%%%%%%%%%%%%%%%%% MACRO DEFINITIONS%%%%%%%%%%%%%%%%%%%%%%%%%%%%%%%%%%%%%%%%%%

%============================== Document macro =======================================
\newcommand{\doc}[1]{%
                    \begin{document}
                    #1
                    \end{document}}
%=====================================================================================                    
                    
%======================= NICER ITEMIZE/ENUMERATE SYNTAX ==============================
\newcommand{\ol}[1]{\begin{enumerate}#1\end{enumerate}}
\newcommand{\ul}[1]{\begin{itemize}#1\end{itemize}}
\newcommand{\li}[1]{\item {#1}}
\newlength{\shiftwidth}
\setlength{\shiftwidth}{3em}
\newcommand{\info}[1]{\par\hspace*{#1\shiftwidth}$\bullet$\quad}
%=====================================================================================

%========================== NICER IMAGE INCLUDE SYNTAX================================
\newcommand{\fig}[2]{%
						 \begin{figure}
						     \centering\includegraphics[#1]{#2}
						     \label{fig:#2}
						 \end{figure}}
\newcommand{\figc}[3]{%
						  \begin{figure}
						      \centering\includegraphics[#1]{#2}
						      \caption{\footnotesize #3}
						      \label{fig:#2}
						  \end{figure}}
%=====================================================================================

%============================= LAZY MULTIICOL ENVIRONMENT ============================
\newcommand{\mcol}[2]{%
                     \begin{multicols}{#1}
                     #2
                     \end{multicols}
                     }
%=====================================================================================

%============================== Shorter vspace ======================================
\newcommand{\vs}[1]{%
                   \vspace{#1}
                   }
\newcommand{\vst}[1]{%
                    \vspace*{#1}
                    }
%===================================================================================

%============================== Lazy frame =========================================
\newcommand{\f}[1]{%
                  \begin{frame}
                  #1
                  \end{frame}
                  }
\newcommand{\ft}[2]{%
				   \begin{frame}{#1}
				   #2
				   \end{frame}
				   }
%=================================================================================== 

%========================== Simple table ===========================================
\newcommand{\tab}[2]{%
				    \begin{table}[!htbp]
				        \begin{tabular}{#1}\hline
				            #2
				        \end{tabular}
				    \end{table}}
					                   
%%%%%%%%%%%%%%%%%%%%%%%%%%%%%%%%%%%%%%%%%%%%%%%%%%%%%%%%%%%%%%%%%%%%%%%%%%%%%%%%%%%%%%
\begin{document}

%%%%%%%%%%%%%%%%%%%%%%%%%%%%%%%%%%%%%%%%%%%%%%%%%%%%%%
\f{\titlepage}
%%%%%%%%%%%%%%%%%%%%%%%%%%%%%%%%%%%%%%%%%%%%%%%%%%%%%%

%%%%%%%%%%%%%%%%%%%%%%%%%%%%%%%%%%%%%%%%%%%%%%%%%%%%%%
\f{\tableofcontents}
%%%%%%%%%%%%%%%%%%%%%%%%%%%%%%%%%%%%%%%%%%%%%%%%%%%%%%
\section{Introduction}
%%%%%%%%%%%%%%%%%%%%%%%%%%%%%%%%%%%%%%%%%%%%%%%%%%%%%%
\ft{Introduction}{%
Last time we met the following objectives were stated:\\
\vspace{.76cm}

\tab{|l | l|}{
Goal & Status \\ \hline
Basic ion exchange experiments for complex quats & Achieved \\ 
Temperature vs diffusion coefficent studies & Achieved (mostly) \\ 
Polymer film studies & In progress \\ 
Provision of acrylamide beads to IP & In progress \\ \hline} \vspace{0.4cm}

\ul{%
	\li{Ion exchange studies themseles are now essentially complete, tough NMR analysis has been delayed}
	\li{Polymr film studies are ongoing, and provision of acrylamide beads to IP will be done as soon as time permits}}}
%%%%%%%%%%%%%%%%%%%%%%%%%%%%%%%%%%%%%%%%%%%%%%%%%%%%%%

\section{Recap: the ion exchange experiment}

%%%%%%%%%%%%%%%%%%%%%%%%%%%%%%%%%%%%%%%%%%%%%%%%%%%%%%
\ft{Recap: the ion exchange experiment}{%
\ul{%
	\li{The ion exchange experiment can be summarized as follows}
}
\mcol{2}{
\fig{scale=0.22}{figs/ionexchange.pdf}
\fig{trim={3cm, 19cm, 6cm, 4.5cm}, clip, scale=0.4}{drafts/exchangeprofiles.pdf}
\vspace{-.30cm}
\ul{
	\setlength{\itemsep{0.2em}}
	\setlength{\topsep{0.1pt}}
	\li{We opt to measure the contents of the ion exchange resin}
	\li{We can measure the sodium ion content (flame photometry) or quaternary ammonium ion content (qNMR spectroscopy)}}}}
%%%%%%%%%%%%%%%%%%%%%%%%%%%%%%%%%%%%%%%%%%%%%%%%%%%%%%

%%%%%%%%%%%%%%%%%%%%%%%%%%%%%%%%%%%%%%%%%%%%%%%%%%%%%%
\ft{Particle size analysis}{
\fig{width=.85\textwidth}{figs/imageAnalysis.PNG}}
%%%%%%%%%%%%%%%%%%%%%%%%%%%%%%%%%%%%%%%%%%%%%%%%%%%%%%

%%%%%%%%%%%%%%%%%%%%%%%%%%%%%%%%%%%%%%%%%%%%%%%%%%%%%%

\ft{Model fitting}{
We now examine which model best fits the ion exchange data \\ 

\tab{|l l|}{
Model Author & Model expression \\ \hline \rule{0pt}{3ex}
Boyd (Sphere) & $F = 1 - \frac{6}{\pi^2} \sum^\infty_1 \frac{1}{n^2}e^{-D_i\pi^2n^2tr^{-2}}$ \\ \rule{0pt}{6ex}
Boyd (Slab) & $F = 1 - \frac{8}{\pi^2}\sum^\infty_1\frac{1}{(2n-1)^2}e^{-(2n-1)^2\pi^2D_it4r^-2_0}$ \\ \rule{0pt}{6ex} 
Diffusion in a bounding film & $log(1-F) = \frac{(\frac{3D_i}{r_0(\Delta r_0)\kappa})}{2.303}t$ \\ \rule{0pt}{6ex}
Hellferich & $F(t) = \sqrt{1-e^{\pi^2f_1(\alpha)t + f_2(\alpha)t^2 + f_3(\alpha)t^3}}$ \\ 
\hline}
%
\ul{
	\setlength{\topsep}{0pt}
	\setlength{\itemsep}{0.2em}
	\li{don't whine}
}}
%%%%%%%%%%%%%%%%%%%%%%%%%%%%%%%%%%%%%%%%%%%%%%%%%%%%%%
\section{Ion exchange data}
%%%%%%%%%%%%%%%%%%%%%%%%%%%%%%%%%%%%%%%%%%%%%%%%%%%%%%
\ft{$R\bar{NMe_4} + Na^+ \leftrightarrow R\bar{Na} + NMe_4^+$ exchange}{}

\end{document}